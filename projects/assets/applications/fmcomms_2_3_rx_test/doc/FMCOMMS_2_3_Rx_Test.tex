\iffalse
This file is protected by Copyright. Please refer to the COPYRIGHT file
distributed with this source distribution.

This file is part of OpenCPI <http://www.opencpi.org>

OpenCPI is free software: you can redistribute it and/or modify it under the
terms of the GNU Lesser General Public License as published by the Free Software
Foundation, either version 3 of the License, or (at your option) any later
version.

OpenCPI is distributed in the hope that it will be useful, but WITHOUT ANY
WARRANTY; without even the implied warranty of MERCHANTABILITY or FITNESS FOR A
PARTICULAR PURPOSE. See the GNU Lesser General Public License for more details.

You should have received a copy of the GNU Lesser General Public License along
with this program. If not, see <http://www.gnu.org/licenses/>.
\fi

%----------------------------------------------------------------------------------------
% Update the docTitle and docVersion per document
%----------------------------------------------------------------------------------------
\def\docTitle{OpenCPI\\ fmcomms\_2\_3\_rx.rcc Test App Guide}
\def\docVersion{1.4}
%----------------------------------------------------------------------------------------
\input{../../../../../doc/av/tex/snippets/LaTeX_Header.tex}
\date{Version \docVersion} % Force date to be blank and override date with version
\title{\docTitle}
\lhead{fmcomms\_2\_3\_rx.rcc Test App Guide}
%----------------------------------------------------------------------------------------
%\usepackage[T1]{fontenc} % http://tex.stackexchange.com/a/181119
\usepackage{graphicx}
\graphicspath{ {figures/} }
\usepackage{textcomp}
\usepackage{listings}

\begin{document}
\maketitle
%\thispagestyle{fancy}

\section{Description}
This application is intended to perform a hardware-in-the-loop test of the fmcomms\_2\_3\_rx.rcc worker. It tests to ensure that the default values and the expected possible min/max values are applied successfully for the various RF frontend parameters. It also tests that fmcomms\_2\_3\_rx.rcc property writes do not override the fmcomms\_2\_3\_tx.rcc worker property writes when both exist in the OAS. Note that RX data fidelity is not verified within this test.

\section{Hardware Portability}
This application is intended to test fmcomms\_2\_3\_rx.rcc, which is by design specific to the FMCOMMS2/3 cards. There is nothing about the application that precludes use of any particular HDL platform, assuming that platform includes an FMC slot on which an FMCOMMS2/3 card may be used.

\section{Execution}
\subsection{Prerequisites}
The following must be true before application execution:
\begin{itemize}
  \item An OpenCPI platform is available w/ an FMCOMMS2 or FMCOMMS3 card plugged into its FMC slot.
  \item The following assets are built for the HDL/RCC platform which correspond to the intended HDL/RCC runtime containers, and their build artifacts (FPGA bitstream file/shared object files) are contained within the directory list of the OCPI\_LIBRARY\_PATH environment variable.
  \begin{itemize}
    \item \verb+empty+ assembly with one of the *\verb+fmcomms_2_3+* containers for the desired HDL platform
    \item \verb+fmcomms_2_3_rx.rcc+ for all \verb+TYPE_p+ configurations (both fmcomms2 and fmcomms3)
    \item \verb+fmcomms_2_3_tx.rcc+ for all \verb+TYPE_p+ configurations (both fmcomms2 and fmcomms3)
    \item \verb+ad9361_config_proxy.rcc+
  \end{itemize}
  \item The application itself (\verb+fmcomms_2_3_rx_test+) must be built.
  \item The current directory is the applications/fmcomms\_2\_3\_rx\_test directory.
\end{itemize}
\subsection{Command(s)}
The full test is run with the following command:
\lstset{language=bash, backgroundcolor=\color{lightgray}, columns=flexible, breaklines=true, prebreak=\textbackslash, basicstyle=\ttfamily, showstringspaces=false,upquote=true, aboveskip=\baselineskip, belowskip=\baselineskip}
\begin{lstlisting}
./<target-dir>/fmcomms_2_3_rx_test
\end{lstlisting}
A software-only test can be run which performs testing on the fmcomms\_2\_3\_rx.rcc software calculation routines (no hardware actuation).
\begin{lstlisting}
./<target-dir>/fmcomms_2_3_rx_test swonly
\end{lstlisting}
\section{Verification}
An application exit status of 0 indicates success, and non-zero indicates failure. Either PASSED or FAILED will also be printed to the screen.

\section{Troubleshooting}
The application will occasionally fail with the following printed to screen:
\begin{lstlisting}
      variable: actual_ad9361_config_proxy_val, expected value: 2083340,  actual value: 2083340 EXPECTED
Calibration TIMEOUT (0x16, 0x10)
Exception thrown: Worker ad9361_config_proxy produced error during execution: ad9361_set_tx_sampling_freq() returned: -110
FAILED
\end{lstlisting}
The fmcomms\_2\_3\_rx.rcc endpoint proxy controls the ad9361\_config\_proxy.rcc device proxy which wraps the ADI No-OS library for SPI command/control of the AD9361. A known defect of the AD9361 hardware/No-OS library is that the AD9361 hardware will occasionally fail to calibrate when No-OS sets low sample rate values (such as 2083340 sps in example above), resulting in a calibration timeout. When this occurs, No-OS prints to the screen:
\begin{lstlisting}
Calibration TIMEOUT (0x16, 0x10)
\end{lstlisting}
Neither of the ad9361\_config\_proxy.rcc/fmcomms\_2\_3\_rx.rcc workers yet implement a mechanism for overcoming this AD9361/No-OS shortcoming.


\end{document}
