\iffalse
This file is protected by Copyright. Please refer to the COPYRIGHT file
distributed with this source distribution.

This file is part of OpenCPI <http://www.opencpi.org>

OpenCPI is free software: you can redistribute it and/or modify it under the
terms of the GNU Lesser General Public License as published by the Free Software
Foundation, either version 3 of the License, or (at your option) any later
version.

OpenCPI is distributed in the hope that it will be useful, but WITHOUT ANY
WARRANTY; without even the implied warranty of MERCHANTABILITY or FITNESS FOR A
PARTICULAR PURPOSE. See the GNU Lesser General Public License for more details.

You should have received a copy of the GNU Lesser General Public License along
with this program. If not, see <http://www.gnu.org/licenses/>.
\fi

%----------------------------------------------------------------------------------------
% Update the docTitle and docVersion per document
%----------------------------------------------------------------------------------------
\def\docTitle{OpenCPI\\ TX Event Test App Guide}
\def\docVersion{1.4}
%----------------------------------------------------------------------------------------
\input{../../../../../doc/av/tex/snippets/LaTeX_Header.tex}
\date{Version \docVersion} % Force date to be blank and override date with version
\title{\docTitle}
\lhead{TX Event Test App Guide}
%----------------------------------------------------------------------------------------
%\usepackage[T1]{fontenc} % http://tex.stackexchange.com/a/181119
\usepackage{graphicx}
\graphicspath{ {figures/} }
\usepackage{textcomp}
\usepackage{listings}

\begin{document}
\maketitle
%\thispagestyle{fancy}
\newpage
	\begin{center}
	\textit{\textbf{Revision History}}
		\begin{table}[H]
		\label{table:revisions} % Add "[H]" to force placement of table
			\begin{tabularx}{\textwidth}{|c|X|l|}
			\hline
			\rowcolor{blue}
			\textbf{Revision} & \textbf{Description of Change} & \textbf{Date} \\
		    \hline
		    v1.4 & Initial Release & 8/2018 \\
			\hline
			\end{tabularx}
		\end{table}
	\end{center}

\newpage
\tableofcontents
\pagebreak

\section{Description}
This application directory containts several OAS files which allow for the testing of the \verb+tx_event+ protocol using the \verb+qdac+ component's \verb+event_in+ port in multiple scenarios include both \verb+event_in+ port connected and unconnected. The application transmits a single tone at a frequency set via the \verb+tx+ component's \verb+frequency_MHz+ property in each OAS. OAS files exist for each supported hardware setup for two different behaviors: 1) transmitted tone is toggled (on/off) at a rate of once per second (\verb+event_in+ port is connected), and 2) transmitted tone is constant over duration of application (\verb+event_in+ port is unconnected). The application duration is set via ocpirun ... -t $<$duration\_sec$>$ ...).

\section{Example usage}

\subsection{Example 1 - Zedboard/FMCOMMS3 w/ SMA TX1A connected to spectrum analyzer w/ 2.4 GHz visible}
\lstset{language=bash, columns=flexible, breaklines=true, prebreak=\textbackslash, basicstyle=\ttfamily, showstringspaces=false,upquote=true, aboveskip=\baselineskip, belowskip=\baselineskip}
\begin{lstlisting}
OCPI_LIBRARY_PATH=../../hdl/assemblies/data_src_to_dac_test_tx_event/:../../hdl/:../../components/:$OCPI_LIBRARY_PATH ocpirun -t 10 tx_event_test_toggle_fmcomms3_control_plane_100_MHz.xml
\end{lstlisting}
Spectrum analyzer is used to observe tone at 2.4 GHz toggle on/off once a second for 10 sec. \\

\subsection{Example 2 - Zedboard/FMCOMMS3 w/ SMA TX1A connected to spectrum analyzer w/ 2.4 GHz visible}
\lstset{language=bash, columns=flexible, breaklines=true, prebreak=\textbackslash, basicstyle=\ttfamily, showstringspaces=false,upquote=true, aboveskip=\baselineskip, belowskip=\baselineskip}
\begin{lstlisting}
OCPI_LIBRARY_PATH=../../hdl/assemblies/data_src_to_dac/:../../hdl/:../../components/:$OCPI_LIBRARY_PATH ocpirun -t 10 tx_event_test_const_tone_fmcomms3.xml
\end{lstlisting}
Spectrum analyzer is used to observe constant tone at 2.4 GHz for 10 sec. \\

\end{document}
